\begin{song}{Muzeum}{Jarek Nohavica}

\songpart{1.}
\begin{guitar}
^ [C] Ve Slezském muzeu [G]v oddělení  [Ami]třetihor je bílý [F]krokodýl\\
^ a [C]medvěd a liška a [G7]kamenní   [C]trilobiti,[G7]\\
^ [C]chodí se tam jen tak, co [G]noha nohu [Ami]mine, abys viděl, jak ten [F]život plyne\\
^ [C]jaké je to všechno [G7]pomíjivé   [C]živobytí.\\
^ [F]Pak vyjdeš do parku a [Csus4]celou noc se [F]touláš noční Opavou a [B]opájíš se\\
^ [F]představou, jaké to [C]bude [F]v ráji.\\
^ V [C]5:35 jednou z [G]pravidelných [Ami]linek, sedm zastávek [F]do Kateřinek,\\
^ [C]ukončete nástup, [G7]dveře se [C]zavírají.\\
\end{guitar}

\songpart{2.}
\begin{guitar}
Budeš-li poslouchat a nebudeš-li odmlouvat, složíš-li svoje \\
maturity, vychováš pár dětí a vyděláš dost peněz,\\
můžeš se za odměnu svézt na velkém kolotoči, dostaneš \\
krásnou knihu s věnováním, zaručeně.\\
A ty bys chtěl plout na hřbetě krokodýla po řece Nil a volat: "Tutanchámon, \\
Vivat, Vivat!" po egyptském kraji.\\
V 5:35 jednou z pravidelných linek, sedm zastávek do Kateřinek,\\
ukončete nástup, dveře se zavírají.\\
\end{guitar}

\songpart{3.}
\begin{guitar}
Pionýrský šátek uvážeš si kolem krku, ve Valtické poručíš si \\
čtyři deci rumu a utopence k tomu.\\
Na rozbitém stole na ubruse píšeš svou rýmovanou Odysseu,\\
nežli přijde někdo, abys šel už domů.\\
Ale není žádné doma jako není žádné venku, není žádné venku, to jsou jenom \\
slova, která obrátit se dají.\\
V 5:35 jednou z pravidelných linek, sedm zastávek do Kateřinek,\\
ukončete nástup, dveře se zavírají.\\
\end{guitar}

\songpart{4.}
\begin{guitar}
Možná si k tobě někdo přisedne, a možná to bude zrovna muž, který \\
osobně znal Egypťana Sinuheta.\\
Dřevěnou nohou bude vyťukávat do podlahy rytmus \\
metronomu, který tady klepe od počátku světa.\\
Nebyli jsme, nebudem a nebyli jsme, nebudem a co budem, až nebudem,\\
jen navezená mrva v boží stáji.\\
V 5:35 jednou z pravidelných linek, sedm zastávek do Kateřinek,\\
ukončete nástup, dveře se zavírají.\\
\end{guitar}

\begin{guitar}
sólo\\
\end{guitar}

\songpart{5.}
\begin{guitar}
Žena doma pláče a děti doma pláčí, pes potřebuje venčit\\
a stát potřebuje daň z přidané hodnoty.\\
A ty si koupíš krejčovský metr a pak nůžkama odstříháváš\\
pondělí, úterý, středy, čtvrtky, pátky, soboty.\\
V neděli zajdeš do Slezského muzea podívat se na vitrínu,\\
kterou tam pro tebe už mají.\\
V 5:35 jednou z pravidelných linek, sedm zastávek do Kateřinek,\\
ukončete nástup, dveře se zavírají. \\
\end{guitar}

\begin{guitar}
sólo\\
\end{guitar}
\end{song}
